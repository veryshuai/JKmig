\documentclass{article}
\usepackage{amsmath}
\usepackage{amssymb}
\usepackage{amsthm}
\usepackage{bbm} %for indicator function
\usepackage[margin = 1in]{geometry}

\title{Initial Attempt at Two-Country Lucas-Moll}
\author{David Jinkins}

%front matter for theorem defs
\newtheorem{theorem}{Theorem}[section]
\newtheorem{lemma}[theorem]{Lemma}
\newtheorem{proposition}[theorem]{Proposition}
\newtheorem{corollary}[theorem]{Corollary}
%end theorem defs
			    
\begin{document}
\maketitle

There are two countries, a rich country and a poor country.  
For now, let the populations of the two countries be the same, with unit mass in each country.
There is a distribution of ``production cost'' for nationals of each country.
In the rich country, this distribution is called $F$, and in the poor country it is called $G$.  
Assume that the $G$ first-order stochastically dominates $F$, and for simplicity of notation assume that both distributions are absolutely continuous.
An individual of cost level $z$ produces and consumes $z^{-\theta}$ in each period.
The only decision that an individual makes is where to live.
An individual chooses to live in either home $h$ or foreign $f$.
Living in foreign is costly.
Individuals living in foreign have their flow production reduced by a factor of $1+B$ (for barrier).
Preferences over consumption and location choice at time $\tau$ are ordered:
\begin{equation}
  U(c) = \int_\tau^\infty e^{-\rho (t-\tau)} c(t) dt  
\end{equation}

As in LM, people randomly run into each other.  
Meetings are proportional to population, and happen with poisson rate $\alpha$.  
We are considering Nash equilibria in which there is a (possibly infinite) cut-off cost above which all poor country nationals migrate to the rich country. 
Denote such a cut-off cost by $z_m(t)$.

Taking the cut-off as given, the cost distributions evolve according to the following rules:
\begin{align}
  \label{fevol}
  \frac{\partial f(z,t)}{\partial t} &= \alpha f(z,t) \Big( -F(z,t) + \left(1-F(z,t)\right) - \big(G(z,t) - G(\min\{z_m(t),z\},t) \big) \Big) \\
  \label{gevol}
  \frac{\partial g(z,t)}{\partial t} &= 
  \begin{cases} 
    \alpha g(z,t) \Big( -G(z,t) + \big(G(z_m(t),t) - G(z,t) \big) \Big) +\alpha f(z,t) \big(1-G(z_m(t),t)\big) & \mbox{if } z<z_m(t) \\
    \alpha g(z,t) \Big( -\big(G(z,t)-G(z_m(t),t)\big) + \big(1-G(z,t)\big) - F(z,t) \Big) &\mbox{if } z>z_m(t)
  \end{cases}
\end{align}

Since the cost distribution is better in the rich country, rich country nationals always choose to live at home.  
Let $\Delta V(y,t) := V(y,t)-V(z,t)$. 
A poor country national's value function satisfies:  
\begin{align}
  \label{valfun}
  \rho V(z,t) &= \frac{\partial V(z,t)}{\partial t} + \max\{z^{-\theta} +  \alpha \int_0^{\min \{z,z_m(t)\}} \Delta V g(y,t) dy, \nonumber  \\
  & \frac{1}{1+B} z^{-\theta}+ \alpha \int_0^z \Delta V f(y,t) dy + \alpha \int_{z_m(t)}^{\max \{z,z_m(t)\}}\Delta V g(y,t)dy \}
\end{align}
A cut-off point, $z_m(t)$ satisfies:
\begin{equation}
  \label{cuteq}
  z_m(t)^{-\theta} + \alpha \int_0^{z_m(t)} \Delta V g(y,t) dy = \frac{1}{1+B} z_m(t)^{-\theta} + \alpha \int_0^{z_m(t)} \Delta V f(y,t)dy
\end{equation}
In words, \eqref{cuteq} says that the expected discounted lifetime utility from living at home is equal to the expected discounted lifetime utility from moving abroad for a poor country individual of cost type $z_m(t)$.
\begin{lemma}
  \label{uniqcut}
 If it exists, the cut-off cost is unique. 
\end{lemma}
\begin{proof}
  We drop time subscripts which are irrelevant in this proof.
  Let $z$ denote a cut-off cost.
  Rearranging \eqref{cuteq}, we can write:
  \begin{equation}
    \label{cutproof1}
    \alpha \int_0^{z} \Delta V \left(f(y)-g(y)\right)dy = \frac{B}{1+B}z^{-\theta}
  \end{equation}
  Since $\theta > 0$, the right-side of \eqref{cutproof1} is a strictly decreasing function of $z$, as long as $B>0$.  
  Thinking of the left-hand side of \eqref{cutproof1} as a function of $z$, we write:
  \begin{align}
    \label{cutproof2}
    \frac{d LHS(z)}{d z} &= \alpha \left(V(z)-V(z)\right)\left(f(z)-g(z)\right) + -V_z(z) \int_0^z \left(f(y)-g(y)\right) dy \nonumber \\
    &= -V_z(z) \int_0^z \left(f(y)-g(y)\right) dy
  \end{align}
  Since there is no benefit to having a higher cost, $V$ must be a weakly decreasing function of $z$.  That $G$ stochastically dominates $F$ then implies that \eqref{cutproof2} is weakly positive.  It follows that there can only be a single crossing of the LHS and RHS of \eqref{cutproof1}, so there is a unique cut-off cost if it exists.
\end{proof}

An equilibrium in this setup is very similar to that in LM.  
An equilibrium, given initial distributions $F(z,0)$ and $G(z,0)$, is a triple $(F,G,V)$ of functions on $\mathbb{R}_+^2$ and a function $z_m$ on $\mathbb{R}_+$ such that: 
\begin{enumerate}
  \item for all $t$, $F(\cdot,t)$ and $G(\cdot,t)$ are probability distributions with $G$ first order stochastically dominating $F$, 
  \item given $z_m$, $F$ and $G$ satisfy \eqref{fevol} and \eqref{gevol},
  \item given $F$ and $G$, $V$ satisfies \eqref{valfun},
  \item $z_m$ is consistent with maximization of \eqref{valfun} given $F$ and $G$. 
\end{enumerate}

\section{Balanced Growth Path}
Although I'm not sure whether this is the most interesting place to start, we follow LM in defining a balanced growth path.
In the balanced growth path, the rich and poor country will grow at the same rate.
To my surprise, this is not very far off from what we see in the data if we look at growth rates in per capita income.
Check this link out: www.bit.ly/N1lnMg.
There does not seem to be a negative relationship between GDP per capita and growth in GDP per capita.

A balanced growth path is a numbers $\gamma$ and $\bar{z}$, and three functions $\phi,\psi$, and $v$, such that:
\begin{enumerate}
  \item $f(z,t) = e^{\gamma t} \phi(z e^{\gamma t})$,
  \item $g(z,t) = e^{\gamma t} \psi(z e^{\gamma t})$,
  \item $V(z,t) = e^{\theta \gamma t} v(ze^{\gamma t})$,
  \item $z_m(t) = e^{-\gamma t} \bar{z}$.
\end{enumerate}
As in LM, the specifications of $f$ and $g$ are so that quantiles of the distributions $F$ and $G$ shrink at the same rate $\gamma$.
The cut-off is then defined so that it remains at the same quantile of $G$, i.e. $G(z_m(t) e^{\gamma t}) = G(\bar{z})$ for all $t$.
The value functions form follows because it keeps \eqref{valfun} stationary given the other balanced growth path conditions.\footnote{This value function part needs to be checked carefully.}

As in LM, some nice relationships come out of the balanced growth path definitions.

\end{document}


